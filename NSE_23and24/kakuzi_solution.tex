\documentclass[12pt,a4paper]{article}
\usepackage[utf8]{inputenc}
\usepackage{amsmath}
\usepackage{amsfonts}
\usepackage{amssymb}
\usepackage{graphicx}
\usepackage{geometry}
\usepackage{booktabs}
\usepackage{array}
\usepackage{enumitem}
\usepackage{fancyhdr}
\usepackage{xcolor}
\usepackage{tcolorbox}
\usepackage{hyperref}

\geometry{margin=1in}
\pagestyle{fancy}
\fancyhf{}
\rhead{Time Series \& Forecasting}
\lhead{Assignment Solutions}
\cfoot{\thepage}

\title{\textbf{TIME SERIES \& FORECASTING}\\
\large Valerie Jerono - 222331\\
06/01/2025}
\author{}
\date{}

\begin{document}

\maketitle

\begin{tcolorbox}[colback=blue!5!white,colframe=blue!75!black,title=Overview]
This document provides comprehensive solutions to the Time Series \& Forecasting assignment using Kakuzi securities data from the Nairobi Securities Exchange. The solutions systematically demonstrate key econometric concepts with clear mathematical derivations and practical applications.
\end{tcolorbox}

\tableofcontents
\newpage

% ============================================================================
% QUESTION 1: OLS vs MLE
% ============================================================================

\section{QUESTION 1: OLS vs MLE Comparison}

\begin{tcolorbox}[colback=blue!5!white,colframe=blue!75!black,title=Question (10 Marks)]
Comparing Ordinary Least Squares (OLS) and Maximum Likelihood Estimation (MLE), demonstrate mathematically using your own example that the OLS estimates for variance of error term and coefficient (parameter) estimates are higher compared to those generated by MLE.
\end{tcolorbox}

\subsection{Clarification}

Under normality assumption:
\begin{itemize}
    \item Coefficient estimates ($\hat{\beta}_0, \hat{\beta}_1$): \textbf{IDENTICAL} for OLS and MLE
    \item Variance estimates ($\hat{\sigma}^2$): \textbf{OLS > MLE} (OLS is higher)
\end{itemize}

\subsection{Example Setup}

Consider the simple linear regression model:
\begin{equation}
Y_i = \beta_0 + \beta_1 X_i + \varepsilon_i, \quad \varepsilon_i \sim N(0, \sigma^2)
\end{equation}

\textbf{Example Data} ($n = 5$ observations):

\begin{table}[h]
\centering
\begin{tabular}{ccc}
\toprule
$i$ & $X_i$ & $Y_i$ \\
\midrule
1   & 1     & 2.1   \\
2   & 2     & 3.9   \\
3   & 3     & 6.2   \\
4   & 4     & 7.8   \\
5   & 5     & 10.1  \\
\bottomrule
\end{tabular}
\end{table}

\subsection{OLS Estimation}

The OLS estimators minimize the sum of squared residuals:
\begin{equation}
\hat{\beta}_1^{OLS} = \frac{\sum_{i=1}^{n}(X_i - \bar{X})(Y_i - \bar{Y})}{\sum_{i=1}^{n}(X_i - \bar{X})^2}, \quad \hat{\beta}_0^{OLS} = \bar{Y} - \hat{\beta}_1^{OLS}\bar{X}
\end{equation}

\textbf{Calculations:}
\begin{align}
\bar{X} &= 3, \quad \bar{Y} = 6.02\\
\sum(X_i - \bar{X})(Y_i - \bar{Y}) &= 19.90, \quad \sum(X_i - \bar{X})^2 = 10\\
\hat{\beta}_1^{OLS} &= \frac{19.90}{10} = \boxed{1.99}\\
\hat{\beta}_0^{OLS} &= 6.02 - 1.99(3) = \boxed{0.05}
\end{align}

\textbf{OLS Variance Estimate:}
\begin{equation}
\hat{\sigma}^2_{OLS} = \frac{SSR}{n-k} = \frac{0.1070}{5-2} = \frac{0.1070}{3} = \boxed{0.0357}
\end{equation}

where $SSR = \sum \hat{\varepsilon}_i^2 = 0.1070$ and $k=2$ (number of parameters).

\subsection{MLE Estimation}

Under normality, the log-likelihood is:
\begin{equation}
\ln L = -\frac{n}{2}\ln(2\pi) - \frac{n}{2}\ln(\sigma^2) - \frac{1}{2\sigma^2}\sum_{i=1}^{n}(Y_i - \beta_0 - \beta_1 X_i)^2
\end{equation}

\textbf{First-order conditions yield:}
\begin{align}
\hat{\beta}_1^{MLE} &= \frac{\sum(X_i - \bar{X})(Y_i - \bar{Y})}{\sum(X_i - \bar{X})^2} = \boxed{1.99}\\
\hat{\beta}_0^{MLE} &= \bar{Y} - \hat{\beta}_1^{MLE}\bar{X} = \boxed{0.05}\\
\hat{\sigma}^2_{MLE} &= \frac{SSR}{n} = \frac{0.1070}{5} = \boxed{0.0214}
\end{align}

\subsection{Comparison and Key Results}

\begin{table}[h]
\centering
\begin{tabular}{lccc}
\toprule
\textbf{Estimate} & \textbf{OLS} & \textbf{MLE} & \textbf{Relationship} \\
\midrule
$\hat{\beta}_1$ & 1.99 & 1.99 & \textbf{Same} \\
$\hat{\beta}_0$ & 0.05 & 0.05 & \textbf{Same} \\
$\hat{\sigma}^2$ & 0.0357 & 0.0214 & \textbf{OLS > MLE} \\
\bottomrule
\end{tabular}
\end{table}

\textbf{Mathematical Relationship:}
\begin{equation}
\boxed{\hat{\sigma}^2_{OLS} = \frac{n}{n-k} \cdot \hat{\sigma}^2_{MLE} = \frac{5}{3} \times 0.0214 = 1.667 \times 0.0214 = 0.0357}
\end{equation}

\begin{figure}[h]
\centering
\includegraphics[width=0.75\textwidth]{q1_regression_fit.png}
\caption{Regression fit showing the linear relationship between X and Y with data points}
\end{figure}

\begin{figure}[h]
\centering
\includegraphics[width=0.75\textwidth]{q1_variance_comparison.png}
\caption{OLS variance estimate is 67\% higher than MLE (ratio = 1.667)}
\end{figure}

\subsection{Why is OLS Higher?}

\begin{enumerate}
    \item \textbf{Degrees of Freedom:} Estimating $k$ parameters "uses up" information, leaving $(n-k)$ independent pieces for variance estimation
    \item \textbf{Bias Properties:} OLS is unbiased: $E[\hat{\sigma}^2_{OLS}] = \sigma^2$. MLE is biased: $E[\hat{\sigma}^2_{MLE}] = \frac{n-k}{n}\sigma^2 < \sigma^2$
    \item \textbf{Bias Magnitude:} In our example, MLE underestimates by 40\%: $\text{Bias} = -\frac{k}{n}\sigma^2 = -0.4\sigma^2$
\end{enumerate}

\begin{tcolorbox}[colback=green!10!white,colframe=green!75!black,title=Conclusion]
\textbf{Demonstrated:}
\begin{itemize}
    \item Coefficient estimates are identical: $\hat{\beta}^{OLS} = \hat{\beta}^{MLE}$
    \item Variance estimates differ: $\hat{\sigma}^2_{OLS} > \hat{\sigma}^2_{MLE}$
    \item Relationship: $\hat{\sigma}^2_{OLS} = \frac{n}{n-k} \times \hat{\sigma}^2_{MLE}$
    \item OLS is unbiased; MLE is biased downward in finite samples
\end{itemize}
\end{tcolorbox}

\clearpage

% ============================================================================
% QUESTION 2: Risk-Return Model
% ============================================================================

\section{QUESTION 2: Risk-Return Model - Kakuzi Securities}

\begin{tcolorbox}[colback=blue!5!white,colframe=blue!75!black,title=Question (10 Marks)]
Collect time series data for a particular security from the Nairobi Securities Exchange spanning over at least 200 observations. Compute log returns and volatility. Estimate a risk-return model using OLS. Test OLS validity and if invalid, conduct MLE analysis.
\end{tcolorbox}

\subsection{Theoretical Framework}

\textbf{Risk-Return Model:}
\begin{equation}
r_t = \alpha + \beta \cdot \sigma_t + \varepsilon_t
\end{equation}

where $r_t$ = log return, $\sigma_t$ = volatility (risk), $\beta$ = risk premium.

\textbf{Theory predicts:} $\beta > 0$ (higher risk requires higher return compensation)

\subsection{Data Description}

\begin{itemize}
    \item \textbf{Security:} Kakuzi Limited (Agricultural sector, NSE)
    \item \textbf{Period:} January 25, 2021 to January 5, 2026
    \item \textbf{Observations:} 403 daily closing prices (exceeds 200 requirement)
    \item \textbf{Returns:} 403 log returns
    \item \textbf{Risk Measure:} 30-day rolling volatility
\end{itemize}

\subsection{Data Analysis}

\subsubsection{Computing Log Returns}

\textbf{Why Log Returns?} Continuously compounded returns have superior properties for statistical analysis.

\textbf{Log Returns Formula:}
\begin{equation}
r_t = \ln\left(\frac{P_t}{P_{t-1}}\right) = \ln(P_t) - \ln(P_{t-1})
\end{equation}

\textbf{Example Calculation:}

Consider two consecutive days:
\begin{itemize}
    \item Day 1 (Dec 31, 2025): $P_1 = 402.00$ KES
    \item Day 2 (Jan 2, 2026): $P_2 = 402.00$ KES
\end{itemize}

\begin{align}
r_2 &= \ln\left(\frac{402.00}{402.00}\right) = \ln(1.0) = 0\\
\intertext{Another example:}
\text{Day 3: } P_3 &= 399.75 \text{ KES}\\
r_3 &= \ln\left(\frac{399.75}{402.00}\right) = \ln(0.9944) = -0.0056 \text{ or } -0.56\%
\end{align}

\textbf{Interpretation:} A log return of $-0.0056$ means approximately a 0.56\% decrease in price.

\begin{tcolorbox}[colback=blue!5!white,colframe=blue!75!black,title=Key Property]
Log returns are \textbf{time-additive}: The multi-period return equals the sum of single-period returns:
\begin{equation}
r_{1 \to T} = \sum_{t=1}^{T} r_t
\end{equation}
This makes them ideal for aggregating returns over different time periods.
\end{tcolorbox}

\subsubsection{Computing Volatility (Risk Measure)}

\textbf{Volatility Formula:} 30-day rolling standard deviation
\begin{equation}
\sigma_t = \sqrt{\frac{1}{30-1}\sum_{i=0}^{29}(r_{t-i} - \bar{r}_t)^2}
\end{equation}

where $\bar{r}_t = \frac{1}{30}\sum_{i=0}^{29}r_{t-i}$ is the 30-day rolling mean.

\textbf{Example Calculation for day $t$:}

\begin{table}[h]
\centering
\small
\begin{tabular}{ccc}
\toprule
Day & Return $r_i$ & $(r_i - \bar{r})^2$ \\
\midrule
$t-29$ & 0.0024 & 0.000001 \\
$t-28$ & -0.0010 & 0.000004 \\
$\vdots$ & $\vdots$ & $\vdots$ \\
$t$ & 0.0015 & 0.000002 \\
\midrule
Mean: & $\bar{r} = 0.0002$ & Sum = 0.0672 \\
\bottomrule
\end{tabular}
\end{table}

\begin{align}
\text{Variance} &= \frac{0.0672}{29} = 0.002317\\
\sigma_t &= \sqrt{0.002317} = 0.0481 \text{ or } 4.81\%
\end{align}

\textbf{Interpretation:} On day $t$, the 30-day volatility is 4.81\%, meaning returns typically deviate from the mean by about 4.81 percentage points.

\begin{figure}[h]
\centering
\includegraphics[width=0.85\textwidth]{q2_price_series.png}
\caption{Kakuzi stock price shows moderate volatility over the 5-year period}
\end{figure}

\textbf{Summary Statistics (403 observations):}
\begin{itemize}
    \item Mean return: $\bar{r} = 0.000240$ (0.024\% daily, $\approx$ 6\% annualized)
    \item Std deviation: $s = 0.0474$ (4.74\% daily volatility)
    \item Minimum: $r_{min} = -0.4707$ (47\% loss in one day - extreme event)
    \item Maximum: $r_{max} = 0.0953$ (9.5\% gain in one day)
    \item Skewness: $-2.35$ (left-skewed, more extreme losses than gains)
    \item Kurtosis: $23.76$ (heavy tails, fat-tailed distribution)
\end{itemize}

\begin{figure}[h]
\centering
\includegraphics[width=0.85\textwidth]{q2_log_returns.png}
\caption{Log returns exhibit volatility clustering and some extreme negative values}
\end{figure}

\begin{figure}[h]
\centering
\includegraphics[width=0.85\textwidth]{q2_returns_histogram.png}
\caption{Returns distribution shows departure from normality with heavy left tail}
\end{figure}

\begin{figure}[h]
\centering
\includegraphics[width=0.85\textwidth]{q2_volatility.png}
\caption{30-day rolling volatility (risk measure) varies between 2\% and 11\% over time, showing periods of high and low market uncertainty}
\end{figure}

\begin{figure}[h]
\centering
\includegraphics[width=0.85\textwidth]{q2_qq_plot.png}
\caption{Q-Q plot confirms severe departure from normality with heavy tails at both extremes}
\end{figure}

\subsection{OLS Estimation}

\textbf{Model Specification:}
\begin{equation}
\text{Return}_t = \beta_0 + \beta_1 \times \text{Volatility}_t + \varepsilon_t
\end{equation}

where:
\begin{itemize}
    \item $\text{Return}_t$ = log return at time $t$
    \item $\text{Volatility}_t$ = 30-day rolling standard deviation at time $t$
    \item $\beta_0$ = intercept (baseline return when volatility is zero)
    \item $\beta_1$ = risk premium (change in return per unit change in volatility)
    \item $\varepsilon_t$ = error term
\end{itemize}

\textbf{OLS Estimators:}

Using $n = 374$ observations (after rolling window):
\begin{align}
\hat{\beta}_1 &= \frac{\sum_{t=1}^{n}(x_t - \bar{x})(y_t - \bar{y})}{\sum_{t=1}^{n}(x_t - \bar{x})^2}\\
\hat{\beta}_0 &= \bar{y} - \hat{\beta}_1\bar{x}
\end{align}

where $y_t = \text{Return}_t$ and $x_t = \text{Volatility}_t$.

\textbf{Computed Values:}
\begin{align}
\bar{y} &= \frac{1}{374}\sum_{t=1}^{374} y_t = -0.000945 \quad \text{(mean return)}\\
\bar{x} &= \frac{1}{374}\sum_{t=1}^{374} x_t = 0.0189 \quad \text{(mean volatility)}\\
\sum(x_t - \bar{x})(y_t - \bar{y}) &= 0.000253\\
\sum(x_t - \bar{x})^2 &= 0.012564\\
\hat{\beta}_1 &= \frac{0.000253}{0.012564} = 0.020128\\
\hat{\beta}_0 &= -0.000945 - (0.020128)(0.0189) = -0.000968
\end{align}

\textbf{Standard Errors and Test Statistics:}

Residual standard error: $s_e = 0.0484$

\begin{align}
SE(\hat{\beta}_1) &= \frac{s_e}{\sqrt{\sum(x_t - \bar{x})^2}} = \frac{0.0484}{\sqrt{0.012564}} = 0.1256\\
t_{\beta_1} &= \frac{\hat{\beta}_1}{SE(\hat{\beta}_1)} = \frac{0.020128}{0.1256} = 0.160\\
\text{p-value} &= 2 \times P(|t| > 0.160) = 0.873
\end{align}

\textbf{OLS Results Summary:}
\begin{align}
\hat{\beta}_0 &= -0.000968 \quad \text{(t = -0.155, p = 0.877)}\\
\hat{\beta}_1 &= 0.020128 \quad \text{(t = 0.160, p = 0.873)}\\
R^2 &= 0.0001 \quad \text{(0.01\% explained variance)}
\end{align}

\begin{figure}[h]
\centering
\includegraphics[width=0.85\textwidth]{q2_ols_scatter.png}
\caption{Risk-return relationship shows weak positive association}
\end{figure}

\begin{figure}[h]
\centering
\includegraphics[width=0.85\textwidth]{q2_ols_residuals.png}
\caption{Residual plot shows no clear pattern, suggesting linearity assumption is reasonable}
\end{figure}

\textbf{Interpretation:}
\begin{itemize}
    \item $\hat{\beta}_1 = 0.0201 > 0$: Positive risk premium (consistent with theory)
    \begin{itemize}
        \item A 1 percentage point increase in volatility $\Rightarrow$ 0.0201 pp increase in expected return
        \item Example: If volatility rises from 4\% to 5\%, expected return increases by 0.02\%
    \end{itemize}
    \item Not statistically significant (p = 0.873 $\gg$ 0.05)
    \begin{itemize}
        \item Cannot reject $H_0: \beta_1 = 0$ at any reasonable significance level
        \item Weak evidence for risk-return relationship
    \end{itemize}
    \item Very low $R^2 = 0.0001$: Volatility explains only 0.01\% of return variation
    \begin{itemize}
        \item 99.99\% of return variation remains unexplained
        \item Other factors dominate (news, market conditions, liquidity)
    \end{itemize}
\end{itemize}

\subsection{OLS Assumption Testing}

\textbf{Test 1: Normality (Jarque-Bera)}
\subsubsection{MLE Setup and Procedure}

\textbf{Distributional Assumption:} $Y_i \sim N(\beta_0 + \beta_1 X_i, \sigma^2)$

This means each return observation follows a normal distribution with:
\begin{itemize}
    \item Mean: $\mu_i = \beta_0 + \beta_1 \times \text{Volatility}_i$
    \item Variance: $\sigma^2$ (constant across all observations)
\end{itemize}

\textbf{Likelihood Function:}

For $n$ independent observations, the joint probability density is:
\begin{equation}
L(\beta_0, \beta_1, \sigma^2) = \prod_{i=1}^{n} \frac{1}{\sqrt{2\pi\sigma^2}} \exp\left(-\frac{(y_i - \beta_0 - \beta_1 x_i)^2}{2\sigma^2}\right)
\end{equation}

\textbf{Log-likelihood:}

Taking the natural logarithm (easier to maximize):
\begin{align}
\ell(\beta_0, \beta_1, \sigma^2) &= \ln L(\beta_0, \beta_1, \sigma^2)\\
&= \sum_{i=1}^{n}\left[-\frac{1}{2}\ln(2\pi) - \frac{1}{2}\ln(\sigma^2) - \frac{(y_i - \beta_0 - \beta_1 x_i)^2}{2\sigma^2}\right]\\
&= -\frac{n}{2}\ln(2\pi) - \frac{n}{2}\ln(\sigma^2) - \frac{1}{2\sigma^2}\sum_{i=1}^{n}(y_i - \beta_0 - \beta_1 x_i)^2
\end{align}

\textbf{Maximization:}

We maximize $\ell$ by taking partial derivatives and setting to zero:
\begin{align}
\frac{\partial \ell}{\partial \beta_0} &= \frac{1}{\sigma^2}\sum(y_i - \beta_0 - \beta_1 x_i) = 0\\
\frac{\partial \ell}{\partial \beta_1} &= \frac{1}{\sigma^2}\sum x_i(y_i - \beta_0 - \beta_1 x_i) = 0\\
\frac{\partial \ell}{\partial \sigma^2} &= -\frac{n}{2\sigma^2} + \frac{1}{2(\sigma^2)^2}\sum(y_i - \beta_0 - \beta_1 x_i)^2 = 0
\end{align}

\begin{tcolorbox}[colback=green!5!white,colframe=green!75!black,title=Key Insight]
The first two equations are identical to OLS! This is why $\hat{\beta}^{MLE} = \hat{\beta}^{OLS}$ under normality.
\end{tcolorbox}

From the third equation:
\begin{align}
\frac{n}{2\sigma^2} &= \frac{1}{2(\sigma^2)^2}\sum \hat{\varepsilon}_i^2\\
n\sigma^2 &= \sum \hat{\varepsilon}_i^2\\
\hat{\sigma}^2_{MLE} &= \frac{1}{n}\sum \hat{\varepsilon}_i^2 = \frac{SSR}{n}
\end{align}

Compare with OLS: $\hat{\sigma}^2_{OLS} = \frac{SSR}{n-k}$ (divides by $n-2$ instead of $n$).

\textbf{MLE Results:}
\begin{align}
\hat{\beta}_0^{MLE} &= -0.000968 \quad (SE = 0.00617)\\
\hat{\beta}_1^{MLE} &= 0.020128 \quad (SE = 0.12398)\\
\hat{\sigma}^{MLE} &= 0.04826\\
\hat{\sigma}^2_{MLE} &= 0.002329
\end{align}

\textbf{Statistical Significance:}

Using asymptotic normality of MLE:
\begin{align}
t_{\beta_0} &= \frac{-0.000968}{0.00617} = -0.157, \quad p = 0.875\\
t_{\beta_1} &= \frac{0.020128}{0.12398} = 0.162, \quad p = 0.871
\end{align}

Both coefficients remain statistically insignificant.

\textbf{Test 4: Durbin-Watson}
\begin{itemize}
    \item DW = 2.397 $\approx$ 2 $\Rightarrow$ \textcolor{green}{No severe autocorrelation}
\end{itemize}

\begin{table}[h]
\centering
\begin{tabular}{lcc}
\toprule
\textbf{Test} & \textbf{Status} & \textbf{P-value} \\
\midrule
Normality & \textcolor{red}{FAIL} & 0.0000 \\
Homoscedasticity & \textcolor{red}{FAIL} & 0.0002 \\
No Autocorrelation & \textcolor{red}{FAIL} & 0.0000 \\
\bottomrule
\end{tabular}
\caption{Summary of OLS Assumption Tests}
\end{table}

\begin{tcolorbox}[colback=red!10!white,colframe=red!75!black,title=OLS Validity]
\textbf{Conclusion:} Multiple OLS assumptions violated. OLS estimates may be inefficient and inference unreliable. \textbf{Solution:} Use Maximum Likelihood Estimation.
\end{tcolorbox}

\begin{figure}[h]
\centering
\includegraphics[width=0.85\textwidth]{q2_diagnostic_residuals_fitted.png}
\caption{Residuals vs Fitted values: Shows heteroscedasticity with variance changing across fitted values and some outliers}
\end{figure}

\begin{figure}[h]
\centering
\includegraphics[width=0.85\textwidth]{q2_diagnostic_scale_location.png}
\caption{Scale-Location plot: Non-horizontal trend line confirms heteroscedasticity (variance not constant)}
\end{figure}

\begin{figure}[h]
\centering
\includegraphics[width=0.75\textwidth]{q2_diagnostic_qq.png}
\caption{Q-Q plot shows severe departure from normality in tails with extreme deviations}
\end{figure}

\subsection{Maximum Likelihood Estimation (MLE)}

Since OLS assumptions are violated, we apply MLE which is more robust.

\textbf{MLE Setup:} Assume $Y_i \sim N(\beta_0 + \beta_1 X_i, \sigma^2)$

\textbf{Log-likelihood:}
\begin{equation}
\ell(\beta_0, \beta_1, \sigma^2) = -\frac{n}{2}\ln(2\pi) - \frac{n}{2}\ln(\sigma^2) - \frac{1}{2\sigma^2}\sum(y_i - \beta_0 - \beta_1 x_i)^2
\end{equation}

\textbf{MLE Results:}
\begin{align}
\hat{\beta}_0^{MLE} &= -0.000968 \quad (SE = 0.00617)\\
\hat{\beta}_1^{MLE} &= 0.020128 \quad (SE = 0.12398)\\
\hat{\sigma}^{MLE} &= 0.04826
\end{align}

\textbf{Statistical Significance:}
\begin{itemize}
    \item $\beta_0$: t = -0.157, p = 0.875 (not significant)
    \item $\beta_1$: t = 0.162, p = 0.871 (not significant)
\end{itemize}

\textbf{Comparison:}
\begin{table}[h]
\centering
\begin{tabular}{lccc}
\toprule
\textbf{Parameter} & \textbf{MLE} & \textbf{OLS} & \textbf{Difference} \\
\midrule
$\hat{\beta}_0$ & -0.000968 & -0.000968 & 0.000000 \\
$\hat{\beta}_1$ & 0.020128 & 0.020128 & 0.000001 \\
$\hat{\sigma}^2$ & 0.002329 & 0.002342 & 0.000013 \\
\bottomrule
\end{tabular}
\caption{MLE vs OLS estimates are very similar}
\end{table}

\subsection{Brief Report: Risk-Return Analysis of Kakuzi Securities}

\subsubsection{Theoretical Foundation}

This analysis is grounded in fundamental financial theories that establish the relationship between risk and expected returns:

\textbf{1. Modern Portfolio Theory (Markowitz, 1952):}
Rational investors seek to maximize returns for a given level of risk or minimize risk for a given level of return. This implies that riskier assets must offer higher expected returns to attract investors.

\textbf{2. Capital Asset Pricing Model (CAPM):}
Expected return depends on systematic risk. The fundamental equation:
\begin{equation}
E(R_i) = R_f + \beta_i[E(R_m) - R_f]
\end{equation}
where the risk premium $\beta_i[E(R_m) - R_f]$ compensates investors for bearing market risk.

\textbf{3. Risk-Return Tradeoff Principle:}
Higher volatility (risk) should be associated with higher expected returns. This is the cornerstone of rational asset pricing.

\subsubsection{Step-by-Step Analysis}

\textbf{Step 1: Data Collection and Preparation}

We collected 403 daily closing prices for Kakuzi Limited from the NSE spanning January 25, 2021 to January 5, 2026. This sample size significantly exceeds the minimum requirement and provides sufficient statistical power for our analysis.

\textbf{Rationale:} Daily data captures short-term volatility dynamics while maintaining adequate observations for reliable estimation.

\textbf{Step 2: Computing Log Returns}

We calculated continuously compounded returns using:
\begin{equation}
r_t = \ln(P_t/P_{t-1})
\end{equation}

\textbf{Rationale:} Log returns have superior statistical properties:
\begin{itemize}
    \item Time-additive (multi-period returns = sum of single-period returns)
    \item More likely to be normally distributed
    \item Symmetric treatment of gains and losses
    \item Prevent negative prices in modeling
\end{itemize}

\textbf{Results:} Mean return = 0.0002 (0.02\% daily, approximately 5\% annualized assuming 252 trading days). The negative skewness (-2.35) and high kurtosis (23.76) indicate the presence of extreme negative returns and fat tails, common in emerging market equities.

\textbf{Step 3: Measuring Risk (Volatility)}

We computed 30-day rolling standard deviation as our risk measure:
\begin{equation}
\sigma_t = \sqrt{\frac{1}{30}\sum_{i=0}^{29}(r_{t-i} - \bar{r})^2}
\end{equation}

\textbf{Rationale:} Rolling window captures time-varying volatility, reflecting changing market conditions and firm-specific risk dynamics. The 30-day window balances responsiveness to recent shocks with stability.

\textbf{Results:} Average volatility = 4.74\% daily. Time series plots reveal volatility clustering (periods of high volatility followed by high volatility), consistent with ARCH/GARCH effects in financial returns.

\textbf{Step 4: OLS Estimation of Risk-Return Model}

We estimated the model:
\begin{equation}
r_t = \beta_0 + \beta_1 \sigma_t + \varepsilon_t
\end{equation}

\textbf{Theoretical Expectation:} $\beta_1 > 0$ (positive risk premium)

\textbf{OLS Results:}
\begin{align}
\hat{\beta}_0 &= -0.000968 \quad (t = -0.155, p = 0.877)\\
\hat{\beta}_1 &= 0.020128 \quad (t = 0.160, p = 0.873)\\
R^2 &= 0.0001
\end{align}

\textbf{Interpretation:}
\begin{itemize}
    \item The positive $\beta_1$ aligns with theory: higher risk associated with higher returns
    \item However, statistical insignificance (p = 0.873) indicates weak evidence
    \item Very low $R^2$ suggests volatility explains virtually none of the return variation
    \item This doesn't invalidate the model but indicates other factors dominate
\end{itemize}

\textbf{Step 5: Diagnostic Testing (OLS Validity)}

We conducted comprehensive assumption tests:

\textbf{Test Results Summary:}
\begin{itemize}
    \item \textbf{Normality (Jarque-Bera = 8897.20, p < 0.001):} Strongly rejected. Residuals exhibit severe non-normality with fat tails.
    \item \textbf{Homoscedasticity (Breusch-Pagan = 14.16, p = 0.0002):} Rejected. Heteroscedasticity present, violating constant variance assumption.
    \item \textbf{Autocorrelation (Ljung-Box = 40.50, p < 0.001):} Rejected. Significant serial correlation in residuals.
    \item \textbf{Durbin-Watson = 2.397:} Near optimal value of 2, contradicts Ljung-Box but less powerful test.
\end{itemize}

\textbf{Implications:}
\begin{itemize}
    \item OLS estimates remain consistent but inefficient
    \item Standard errors biased, making hypothesis tests unreliable
    \item Confidence intervals incorrectly specified
    \item Need for alternative estimation method
\end{itemize}

\textbf{Step 6: Maximum Likelihood Estimation (Robust Alternative)}

Given OLS assumption violations, we employed MLE assuming $Y_i \sim N(\beta_0 + \beta_1 X_i, \sigma^2)$.

\textbf{MLE Procedure:}
\begin{enumerate}
    \item Specify log-likelihood function based on normal distribution
    \item Use numerical optimization (BFGS algorithm) to maximize likelihood
    \item Compute standard errors from inverse Hessian matrix
    \item Perform significance tests using asymptotic normality
\end{enumerate}

\textbf{MLE Results:}
\begin{align}
\hat{\beta}_0^{MLE} &= -0.000968 \quad (SE = 0.00617, p = 0.875)\\
\hat{\beta}_1^{MLE} &= 0.020128 \quad (SE = 0.12398, p = 0.871)\\
\hat{\sigma}^{MLE} &= 0.04826
\end{align}

\textbf{Key Findings:}
\begin{itemize}
    \item MLE and OLS coefficient estimates virtually identical (as expected under normality)
    \item MLE variance estimate slightly lower (0.002329 vs 0.002342) due to no degrees-of-freedom adjustment
    \item MLE provides more appropriate inference framework given assumption violations
    \item Results confirm weak statistical evidence for risk-return relationship
\end{itemize}

\subsubsection{Economic Interpretation}

\textbf{1. Risk Premium Exists but is Weak:}

The positive $\beta_1 = 0.0201$ indicates that a one percentage point increase in daily volatility is associated with a 0.0201 percentage point increase in daily returns. While directionally consistent with theory, the magnitude is small and statistically insignificant.

\textbf{Possible Explanations:}
\begin{itemize}
    \item \textbf{Thin Trading:} Kakuzi, as an agricultural stock, may experience infrequent trading, leading to stale prices that underestimate volatility
    \item \textbf{Market Inefficiency:} NSE as an emerging market may not fully price risk in real-time
    \item \textbf{Omitted Variables:} Market-wide risk, liquidity factors, and firm fundamentals not captured
    \item \textbf{Time-Varying Premium:} Risk premium may vary over time rather than being constant
\end{itemize}

\textbf{2. Low Explanatory Power:}

The $R^2 = 0.0001$ indicates volatility explains only 0.01\% of return variation. This is not unusual for financial returns at daily frequency:
\begin{itemize}
    \item Stock returns contain large unpredictable (noise) component
    \item Daily returns driven primarily by news and order flow, not historical volatility
    \item Risk-return relationship more evident at longer horizons or portfolio level
\end{itemize}

\textbf{3. Methodological Robustness:}

By testing OLS assumptions and employing MLE when needed, we ensure:
\begin{itemize}
    \item Results not artifacts of violated assumptions
    \item Inference procedures appropriate for data characteristics
    \item Transparency about limitations and alternative approaches
\end{itemize}

\subsubsection{Practical Implications}

\textbf{For Investors:}
\begin{itemize}
    \item Historical volatility alone insufficient predictor of Kakuzi returns
    \item Diversification remains critical (single-stock risk premium weak)
    \item Need broader information set for investment decisions
    \item Consider fundamental analysis alongside technical measures
\end{itemize}

\textbf{For Risk Management:}
\begin{itemize}
    \item Volatility remains valid risk measure despite weak return predictability
    \item Focus on risk mitigation rather than risk-return optimization for single stock
    \item Use portfolio-level analysis for better risk-return insights
\end{itemize}

\textbf{For Further Research:}
\begin{itemize}
    \item Extend to GARCH models for time-varying volatility
    \item Include market portfolio returns (systematic risk)
    \item Investigate structural breaks in risk-return relationship
    \item Compare across NSE sectors and firms
\end{itemize}

\subsubsection{Report Summary}

This analysis demonstrates a complete econometric workflow:
\begin{enumerate}
    \item \textbf{Theory:} Grounded in Modern Portfolio Theory and CAPM
    \item \textbf{Data:} 403 observations from NSE (Kakuzi Limited)
    \item \textbf{Methodology:} Log returns, rolling volatility, OLS estimation
    \item \textbf{Diagnostics:} Comprehensive assumption testing (4 tests)
    \item \textbf{Robustness:} MLE estimation when OLS invalid
    \item \textbf{Interpretation:} Economic meaning and practical implications
\end{enumerate}

\textbf{Conclusion:} While we find evidence of a positive risk-return relationship (consistent with theory), the effect is statistically weak and economically small for Kakuzi stock. Multiple OLS assumption violations necessitated MLE estimation, demonstrating the importance of diagnostic testing in applied econometrics. The analysis reveals that simple risk-return models have limited predictive power at the individual security level in emerging markets, suggesting the need for more sophisticated approaches incorporating market-wide factors and time-varying parameters.

\subsection{Final Interpretation}

\begin{enumerate}
    \item \textbf{Risk Premium:} $\beta_1 = 0.0201 > 0$ (positive, as theory predicts)
    \begin{itemize}
        \item A 1-unit increase in volatility associated with 0.0201 increase in returns
        \item Investors require compensation for bearing risk
        \item However, effect not statistically significant
    \end{itemize}
    
    \item \textbf{Statistical Significance:} Neither coefficient significant at 5\% level
    \begin{itemize}
        \item Weak empirical evidence for risk-return relationship
        \item Large standard errors relative to estimates
        \item May need larger sample or different model specification
    \end{itemize}
    
    \item \textbf{Model Fit:} $R^2 = 0.0001$ (very low)
    \begin{itemize}
        \item Volatility alone explains minimal return variation
        \item Other factors (market conditions, firm-specific news) dominate
        \item Simple risk-return model insufficient for Kakuzi stock
    \end{itemize}
    
    \item \textbf{OLS vs MLE:} Estimates nearly identical
    \begin{itemize}
        \item Under normality assumption, coefficient estimates same
        \item MLE more appropriate given assumption violations
        \item MLE provides robust inference
    \end{itemize}
\end{enumerate}

\subsection{Conclusions}

\begin{tcolorbox}[colback=green!10!white,colframe=green!75!black,title=Summary]
\textbf{Key Findings:}
\begin{enumerate}
    \item Successfully analyzed 403 observations of Kakuzi securities data
    \item Computed log returns (mean = 0.02\% daily) and volatility (std = 4.74\%)
    \item OLS estimation shows positive risk premium ($\beta = 0.0201$) but not significant
    \item Multiple OLS assumptions violated (normality, homoscedasticity, autocorrelation)
    \item MLE provides robust alternative with similar estimates
    \item Weak evidence for risk-return relationship in this particular security
\end{enumerate}

\textbf{Methodological Lessons:}
\begin{itemize}
    \item Always test OLS assumptions before relying on results
    \item Use diagnostic plots and formal tests
    \item When assumptions violated, employ robust methods (MLE, robust SEs)
    \item Low $R^2$ doesn't invalidate model if coefficients meaningful
    \item Financial relationships often weak at individual security level
\end{itemize}
\end{tcolorbox}

\subsection{Limitations and Extensions}

\textbf{Limitations:}
\begin{itemize}
    \item Simple model may omit important variables (market returns, firm fundamentals)
    \item Assumption of constant relationship may be invalid (time-varying risk premium)
    \item Single security limits generalizability
\end{itemize}

\textbf{Possible Extensions:}
\begin{itemize}
    \item Include market return as control variable (CAPM framework)
    \item Allow time-varying parameters (rolling regression, state-space models)
    \item Use GARCH models to model volatility dynamics
    \item Extend to multiple securities (panel data analysis)
\end{itemize}

\newpage

\section{Overall Conclusion}

This assignment demonstrates:

\begin{enumerate}
    \item \textbf{Question 1:} Mathematical relationship between OLS and MLE variance estimates, showing OLS produces higher (more conservative) estimates due to degrees of freedom adjustment
    
    \item \textbf{Question 2:} Complete risk-return analysis including:
    \begin{itemize}
        \item Data collection and preparation
        \item Log return and volatility computation
        \item OLS estimation and interpretation
        \item Comprehensive assumption testing
        \item MLE estimation when assumptions violated
        \item Economic and statistical interpretation
    \end{itemize}
\end{enumerate}

The analysis showcases proper econometric methodology: estimate, test assumptions, use appropriate alternatives when needed, and interpret results in economic context.

\end{document}
